\UseRawInputEncoding
\documentclass{article}
\usepackage[margin=0.5in]{geometry}
\usepackage{hyperref}
\usepackage{ifthen}
\usepackage{multicol}
\usepackage{calc}

\newlength{\remaining}
\newcommand{\titleline}[1]{
\setlength{\remaining}{\textwidth-\widthof{\textsc{#1}}}
\noindent\underline{\textsc{#1}\hspace*{\remaining}}\par}

\begin{document}
    \thispagestyle{empty}
    \begin{center}
         \large{\textbf{Sasha Sheng}} \\
         \href{mailto:sashashengyt@gmail}{sashashengyt@gmail.com} \\
        %  \href{www.sashasheng.com}{www.sashasheng.com} \\
         734-834-7849
    \end{center}
  
    \titleline{\textbf{\large{Work}}}
 	\noindent\textbf{Software/Research Engineer at Facebook} \hfill Oct. 2015 to Nov. 2022 \\ 
 	    \indent Software Engineer at Facebook AI Research \hfill Aug. 2021 to Nov. 2022 \\
 	        \indent\indent As the only Software Engineer on the team, I implemented 3 React \& Flask based demos \\
 	        \indent\indent Led the collaboration between torchserve team, AWS and FAIR teams \\
 	        \indent\indent Led the implementation of the data collection pipeline on MTurk, which is critical to our research \\
 	    \indent Research Engineer on Facebook AI Research \hfill  Oct. 2020 to Aug. 2021 \\
 	        \indent\indent Open Source contributions on \href{https://github.com/facebookresearch/mmf}{Multimodal Framework (MMF)} \\
 	        \indent\indent Led the pytorch lightning integration into MMF \\
 	        \indent\indent Integrated Localized Narratives into MMF \\
 	        \indent\indent Led the collection of Human Adversarial Data through \href{www.dynabench.org}{Dynabench Framework} \\
		\indent Fullstack Software Engineer on AI New Experiences \hfill Mar. 2019 to Oct. 2020 \\
		    \indent\indent Team Lead for Creative AI, prototyping experiences with Generative AI  \\
		    \indent\indent Team Lead for Meeting Assistant, prototyping experiences for remote work \\
		    \indent\indent Project Lead and point of contact in collaboration with ARVR for Assistant Next prototype \\
		\indent Fullstack Software Engineer on News Feed Relevance \hfill Aug. 2017 to Mar. 2019 \\
		    \indent\indent Created the News Feed Labeling Platform Sourcery based on top of an internal tool SRT \\
		    \indent\indent Led a company wide Summit focused on Human Labeling Excellence in August 2018 \\
    	\indent iOS Software Engineer on News Feed Experience \hfill Oct. 2015 to Aug. 2017 \\
		    \indent\indent Built and ran experiments for Topic Feeds product and Explore Feeds product \\
    \noindent\textbf{Software Engineer at Yahoo} \hfill Jan. 2014 to Aug. 2015 \\
    	\indent Tech Lead, Yahoo Mail for iOS \\
                 
    \titleline{\textbf{\large{Education}}}
    	\noindent\textbf{Stanford University, Stanford, USA} \\
	Stanford Ignite Certification, Graduate School of Business \hfill Jun. 2017 to Jul. 2017 \\
	\indent Exposed to core business skills such as marketing, operations, accounting, finance, and design thinking. \\
	\indent Worked in a team to develop a plan for commercializing a new product or service for a new venture. \\
    \noindent\textbf{University of Michigan, Ann Arbor, USA} \\
 	B.S.E., Computer Science Engineering \hfill Sept. 2011 to Dec. 2013 \\
	\indent Courses: Artificial Intelligence, Mobile App Development (iOS), Web Database System, Operating System \\
         \noindent\textbf{Shanghai Jiaotong University, Shanghai, China} \\
         B.S.E., Mechanical Engieering \hfill Sept. 2009 to Aug. 2013 \\
         \indent Courses: Design \& Manufacture, System Dynamics, Heat Transfer, Thermodynamics, Fluid Dynamics \\
    
    \titleline{\textbf{\large{Open Source Contributions}}}
        Multimodal Framework - MMF, Mephisto, Dynabench \\	
       
    \titleline{\textbf{\large{Papers \& Patents}}}
    \noindent{Paper: Human Adversarial Visual Question Answering \hfill 2021 NeurIPS \\
        \indent Contributions: Created an adversarial benchmark (AdVQA) to illustrate the insufficiency of SOTA VQA models.\\
        \indent\indent Led the discussion and ran experiments that compares popular SOTA models on VQA2.0 dataset and AdVQA dataset.  \\
    \noindent{Paper: MUGEN: A Playground for Video-Audio-Text Multimodal Understanding and GENeration} \hfill 2022 ECCV Submission \\
        \indent Contributions: Collected 378,902 video descriptions on MTurk \\
        \indent\indent Human Annotator quality evaluation \\
    \noindent{Patent: Speech Transcription Using Multiple Data Sources} \hfill ID: 16/689,662\\
    \noindent{Defensive Publication (ID14-10403): Crashr: reverse engineer stack trace to auto generate test cases \hfill 2014} \\
    
    \titleline{\textbf{\large{Leadership Experience}}}
        \noindent{Engineering Leader on Test\&Trace \hfill Mar. 2020 to Aug. 2020} \\
        \noindent{Organizer of DeepLearners Meetup} \hfill Jun. 2017 to Jan. 2019 \\
        \noindent{EECS281 Student Instruction Aid, \textbf{Computer Science Department} \hfill Fall 2013, Ann Arbor, MI \\
        \noindent{HKN Recording Secretary Officer, \textbf{Eta Kappa Nu EECS Honor's Society} \hfill Winter 2013, Ann Arbor, MI \\
        \noindent{Transfer Student Leader, \textbf{Michigan Engineering Transfer Student Program}} \hfill Academic Year 2012, Ann Arbor, MI \\
\end{document}
